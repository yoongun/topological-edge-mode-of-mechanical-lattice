\documentclass[../main.tex]{subfiles}
\graphicspath{{\subfix{../images/}}}
\begin{document}

Breaking time-reversal (TR) symmetry known as to cause the energy gap to open
at the Dirac point\cite{lee}. 
By breaking TR, we can introduce topologically protected modes on the material.
These topologically protected modes is one of the most important factor for the unique quantum
phenomena such as a topological magneto-electri effect, an image
magnetic monopole effect, topological Kerr and Faraday rotation, and the
quantum anomalous Hall effect (AHE)\cite{lee}.

Nonetheless many topologically protected modes apears on quantum mechanical
system, Raghu and Haldane proposed nontirivial topological modes are
rather a wave phenomenon than quantum effect by demonstrating
optical analog of quantum Hall effect(QHE) with periodically arrange gyromagnetic rods
\cite{raghu,haldane}.
Many theoretical proofs has been proposed on this field\cite{haldane,wang2,xianyu}.

In this research, we will verify the idea to break the TR symmetry on the
mechanical graphene using coriolis force with non-inertial reference frame of
a rotating system with actual device\cite{kariyado,wang}.
As an experimental device, we introduce two devices. The first one is 1D spring-mass type
chain in which masses are placed on the edge of a circle.

\end{document}