\documentclass[a4paper]{article}
\usepackage{lipsum,multicol}

\title{Visualization of topologically protected modes of mechanical lattices}
\author{Gun Yoon}

\begin{document}
\maketitle
\begin{abstract}
    abstract aaaaaaaaaaaaaaaaaaaaaaaaaaaaaaaaaaaaaaaaaaaaaaaaaaaa
    abstract aaaaaaaaaaaaaaaaaaaaaaaaaaaaaaaaaaaaaaaaaaaaaaaaaaaa
    abstract aaaaaaaaaaaaaaaaaaaaaaaaaaaaaaaaaaaaaaaaaaaaaaaaaaaa
\end{abstract}


\begin{multicols*}{2}

    \section{Introduction}
    Acoustic metamaterials, sometimes called also Elastic metamaterials or Mechanical
    metamaterials, are defined as man-made structures that display acoustic or elastic wave properties not found in nature. This field
    is rapidly growing [1,2,3].

    The recent hottest topic in this field is the use of topological protected modes, which have characteristics of
    robustness and nonreciprocal features when time-reverse symmetry is beaked. It is known that hexagonal
    mechanical lattices have typical band shapes in their dispersion relation which are called Dirac cones. By
    adding small modulations to Dirac cones, we can realize topological protected modes [4].

    By adding rotational modulation into a spring-mass type wave machine, we could break time-reversal
    symmetry by Coriolis force. Then we could realize one-directional topologically protected edge modes [5].
    introduction

    \section{Background}

    \subsection{brillouin zone}

    What is brillouin zone?
    Why using brillouin zone?

    \subsection{Bloch theory}

    What is Bloch theory?

    WHy using Bloch theory?

    \subsection{SSH model}

    \subsection{Chern number}

    \subsection{Topological insulator}

    What is topology?

    What is insulator?

    What is topological insulator?

    What is Hall effect and Quantum hall effect?

    \section{Study of 1 dimensional mechanical lattice}

    \subsection{String}

    \subsection{Circular}

    \section{Result}

    \section{Conculsion}

    \section{Further research}

    \subsection{Topological quantum computing with majorana qubit}


    \section{Reference}


\end{multicols*}
\end{document}